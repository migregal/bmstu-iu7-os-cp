\begin{appendices}
    \chapter{Структура usbmon\_packet}

	\listingfile{usbmon_packet.h}{usbmon-packet}{C}{Структура usbmon\_packet}{}

	\newpage

    \chapter{Структура usb\_device}
	%В Листингах~\ref{lst:migration-1}~--~\ref{lst:migration-2} приведен скрипт проведения миграции базы данных PostgreSQL~версии~14.2 в Docker~контейнере.

	%/inlclude/linux/usb.h

	\listingfile{usb_device.h}{usb-device-1}{C}{Структура usb\_device. Часть 1}{linerange={1-35}}

	\newpage

	\listingfile{usb_device.h}{usb-device-2}{C}{Структура usb\_device. Часть 2}{linerange={37-75}, firstnumber=37}

	\newpage

	\listingfile{usb_device.h}{usb-device-3}{C}{Структура usb\_device. Часть 3}{linerange={76-83}, firstnumber=76}

	\chapter{Структура usb\_device\_descriptor}

	%/include/uapi/linux/usb/ch9.h.
	\listingfile{usb_device_descriptor.h}{usb-device-descriptor}{C}{Структура usb\_device\_descriptor}{linerange={1-18}}

	\chapter{Структура usb\_device\_id}

	%/include/linux/mod_devicetable.h
	\listingfile{usb_device_id.h}{usb-device-id}{C}{Структура usb\_device\_id}{linerange={1-27}}

    \chapter{usermode-helper API}

    \listingfile{umod_helper.h}{umh}{C}{\texttt{usermode-helper API}}{}

	\chapter{Функции идентификации устройств}

	\listingfile{netpmod.c}{usb-identify-1}{C}{Функции идентификации устройств. Часть 1}{linerange={122-151}}

	\newpage

	\listingfile{netpmod.c}{usb-identify-2}{C}{Функции идентификации устройств. Часть 2}{linerange={153-164}, firstnumber=153}
    
\end{appendices}
