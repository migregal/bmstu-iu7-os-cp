\chapter{Аналитический раздел}

\section{Постановка задачи}

В соответствии с заданием на курсовую работу необходимо разработать загружаемый модуля ядра ОС Linux для отключения сетевого оборудования системы при подключении USB–устройства. Для решения данной задачи необходимо:

\begin{itemize}
	\item проанализировать методы обработки событий, возникающих при взаимодействии с USB–устройствами;
	\item проанализировать структуры и функции ядра, предоставляющие информацию о USB–устройствах;
	\item разработать алгоритмы и структуру программного обеспечения;
	\item реализовать программное обеспечение;
	\item исследовать разработанное программное обеспечение.
\end{itemize}

\section{Обработка событий от USB-устройств}

Для обработки событий, возникающих при работе с USB–устройствами, например, таких как подключение или отключение устройства, необходимо узнать о возникновении события и выполнить необходимую обработку после возникновения события.

Далее будут рассмотрены существующие подходы к определению возникновения событий от USB–устройств и выбран наиболее подходящий для реализации в данной работе

\subsection{usbmon}

\subsection{udevadm}

\subsection{Уведомители}

\section{USB-устройства в ядре Linux}

\subsection{Структура usb\_device}

\subsection{Структура usb\_device\_id}