\chapter{Аналитический раздел}

\section{Постановка задачи}

В соответствии с заданием на курсовую работу необходимо разработать загружаемый модуля ядра ОС Linux для отключения сетевого оборудования системы при подключении USB–устройства. Для решения данной задачи необходимо:

\begin{itemize}
	\item проанализировать методы обработки событий, возникающих при взаимодействии с USB–устройствами;
	\item проанализировать структуры и функции ядра, предоставляющие информацию о USB–устройствах;
	\item разработать алгоритмы и структуру программного обеспечения;
	\item реализовать программное обеспечение;
	\item исследовать разработанное программное обеспечение.
\end{itemize}

\section{Обработка событий от USB-устройств}

Для обработки событий, возникающих при работе с USB–устройствами, например, таких как подключение или отключение устройства, необходимо узнать о возникновении события и выполнить необходимую обработку после возникновения события.

Далее будут рассмотрены существующие подходы к определению возникновения событий от USB–устройств и выбран наиболее подходящий для реализации в данной работе

\subsection{usbmon}

\texttt{usbmon}~\cite{usbmon} --- это средство ядра Linux, которое используется для сбора информации о событиях, произошедших на устройствах ввода--вывода, подключенных посредством USB.

\texttt{usbmon} предоставляет информацию о запросах, сделанных драйверами устройств к драйверам хост--контроллера (HCD). В случае, когда драйвера хост--контроллера неисправны, данные, предоставленные \texttt{usbmon}, могут не соответствовать действительным переданным данным.

В настоящее время реализованы два программных интерфейса для взаимодействия с \texttt{usbmon}: 
\begin{itemize}
    \item текстовый --- данный интерфейс устарел, но сохраняется для совместимости;
    \item бинарный --- доступен через символьное устройство в пространстве имен \texttt{/dev}.
\end{itemize}

К особенностям \texttt{usbmon} относятся:

\begin{itemize}
	\item возможность просматривать собранную информацию через специальное ПО (например, \texttt{Wireshark}~\cite{wireshark});
	\item возможность отслеживать события на одном порте USB или на всех сразу;
	\item отсутствие возможности вызова обработчика при возникновении определенного события.
\end{itemize}

При этом, \texttt{usbmon} позволяет отслеживать события, но не позволяет реагировать на них без программной доработки для реализации обработчика.

В листинге \ref{lst:usbmon-packet} представлена структура ответа, полученного после события, случившегося на USB--устройстве (например, подключение к компьютеру).

\subsection{udevadm}

\texttt{udevadm} \cite{udevadm} --- инструмент для управления устройствами \texttt{udev}. Структура \texttt{udev} описана в библиотеке \texttt{libudev} \cite{libudev}, которая не является системной библиотекой Linux. В данной библиотеке представлен программный интерфейс для мониторинга и взаимодействия с локальными устройствами.

При помощи \texttt{udevadm} можно получить полную информацию об устройстве, полученную из его представления в \texttt{sysfs}, чтобы создать корректные правила и обработчики событий для устройства. Кроме того можно получить список событий для устройства, установить наблюдение за ним.

Особенности \texttt{udevadm}:

\begin{itemize}
	\item возможность привязки своего обработчика к событию;
	\item невозможность использования интерфейса в ядре Linux;
\end{itemize}


В листинге \ref{lst:udevadm} представлен пример правила обработки событий, задаваемого с помощью \texttt{udevadm}.

\listingfile{udevadm.sh}{udevadm}{Bash}{Правила \texttt{udevadm}}{}

\subsection{Уведомители}

\section{USB-устройства в ядре Linux}

\subsection{Структура usb\_device}

\subsection{Структура usb\_device\_id}