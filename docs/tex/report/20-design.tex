\chapter{Конструкторский раздел}

\section{IDEF0 последовательность преобразований}

На рисунках \ref{img:idef0-a} и \ref{img:idef0-b} представлена IDEF0 последовательность преобразований.


\imgw{idef0-a}{ht!}{0.8\textwidth}{Нулевой уровень преобразований}

\imgw{idef0-b}{ht!}{0.8\textwidth}{Первый уровень преобразований}

\section{Структура программного обеспечения}

В состав разрабатываемого программного обеспечения входит один загружаемый модуль ядра, который отслеживает подключенные USB--устройства и программно отключает сетевые устройства при наличии недоверенного устройства. 

Недоверенным устройством считается устройство, которое не идентифицируется в соответствии со списком допустимых устройств модуля. Список допустимых устройств задается в исходном коде модуля.

\section{Загружаемый модуль ядра}

Для отслеживания событий подключения и отключения устройств, а так же отслеживания событий ввода с клавиатуры, в модуле ядра размещаются соответствующие уведомители, которые будут зарегистрированы при загрузке модуля и удалены при его выгрузке.

Схема алгоритма загружаемого модуля представлена на рисунке~\ref{img:algo-module}.

\imgw{algo-module}{ht!}{0.3\textwidth}{Схема алгоритма загружаемого модуля}

\section{Обработчик событий от USB}

Для хранения информации о подключенных устройствах будет использован связный список, хранящий информацию об идентификационных данных устройства.

На рисунке~\ref{img:algo-usb} представлен алгоритм работы обработчика событий от USB.

\imgw{algo-usb}{ht!}{0.95\textwidth}{Схема алгоритма обработчика событий от USB}


\section{Обработчик событий от клавиатуры}

На рисунке~\ref{img:algo-kbd} представлен алгоритм работы обработчика событий от клавиатуры.

\imgw{algo-kbd}{ht!}{0.95\textwidth}{Схема алгоритма обработчика событий от клавиатуры}

\section*{Выводы}

Были рассмотрены IDEF0-диаграммы, описывающие разрабатываемый загружаемый модуль ядра, а так же схемы алгоритмов, которые будут использованы в разрабатываемом модуле.