\chapter{Конструкторский раздел}

\section{Последовательность преобразований}

На рисунках \ref{img:idef0-a} и \ref{img:idef0-b} представлена последовательность преобразований.

\imgw{idef0-a}{ht!}{0.8\textwidth}{Нулевой уровень преобразований}

\imgw{idef0-b}{ht!}{0.8\textwidth}{Первый уровень преобразований}

\section{Загружаемый модуль ядра}
\label{sect:lkm-design}

Для отслеживания событий подключения и отключения устройств, а так же отслеживания событий ввода с клавиатуры, в модуле ядра размещаются соответствующие уведомители, которые будут зарегистрированы при загрузке модуля и удалены при его выгрузке.

Схема алгоритма загружаемого модуля представлена на рисунке~\ref{img:algo-module}. Выполняется регистрация уведомителей для обработки событий от USB и клавиатуры.

\imgw{algo-module}{ht!}{0.2\textwidth}{Схема алгоритма загружаемого модуля}

\section{Обработчик событий от USB}
\label{sect:usb-handler-design}

Для хранения информации о подключенных устройствах будет использован связный список, хранящий информацию об идентификационных данных устройства.

На рисунке~\ref{img:algo-usb-in} представлена схема алгоритма работы обработчика события подключения USB--устройства.

В случае подключения USB--устройства производится проверка зарегистрированности данного устройства как доверенного. Если устройство является незарегистрированным, производится отключение сети.

\clearpage

\imgw{algo-usb-in}{ht!}{0.4\textwidth}{Схема алгоритма работы обработчика события подключения USB--устройства}

На рисунке~\ref{img:algo-usb-out} представлена схема алгоритма работы обработчика события удаления USB--устройства.

В случае удаления USB--устройства производится проверка зарегистрированности подключенных устройств как доверенных. Если все устройства являются зарегистрированными, производится подключение сети.


\imgw{algo-usb-out}{ht!}{0.4\textwidth}{Схема алгоритма работы обработчика события удаления USB--устройства}

\section{Обработчик событий от клавиатуры}
\label{sect:kbd-handler-design}

На рисунке~\ref{img:algo-kbd} представлен алгоритм работы обработчика событий от клавиатуры.

\imgw{algo-kbd}{ht!}{0.7\textwidth}{Схема алгоритма обработчика событий от клавиатуры}

В начале проверяется состояние сети устройства -- если сеть не отключена, дальнейшая обработка не требуется. В противном случае, введенный символ сравнивается с очередным символом пароля. В случае, если символы не совпадают, попытка ввода считается неудачной. Если успешно введен весь пароль, производится включение сети.

\section{Структура программного обеспечения}

Составляющие проекта приведены на рисунке~\ref{img:user-kernel}.

\imgw{user-kernel}{ht!}{0.6\textwidth}{Структура ПО}
