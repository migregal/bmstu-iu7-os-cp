\chapter{Технологический раздел}

\section{Выбор языка и среды программирования}

Разработанный модуль ядра написан на языке программирования \texttt{C}~\cite{c-language}. Выбор языка программирования \texttt{С} основан на том, что в настоящий момент большая часть исходного кода ядра Linux, его модулей и драйверов написана на данном языке~\cite{rust-in-linux}.

В качестве компилятора выбран \texttt{gcc}~\cite{gcc}. Выбор обоснован тем, что данный компилятор является предпочтительным для сборки Linux~\cite{build-linux}.

В качестве среды разработки выбрана среда \texttt{Visual Studio Code}~\cite{vscode}.

\section{Хранение информации об отслеживаемых устройствах}

Для хранения информации об отслеживаемых устройствах объявлена структура \texttt{int\_usb\_device}, которая хранит в себе идентификационные данные устройства (\texttt{PID, VID}), а так же указатель на элемент списка.

Структура \texttt{int\_usb\_device}, а так же инициализация списка, в котором букдут храниться данные структуры представлены в листинге~\ref{lst:int-usb}.

\listingfile{netpmod.c}{int-usb}{C}{Структура \texttt{int\_usb\_device}}{linerange={114-120}}

Список отслеживаемых устройств должен модифицироваться при подключении и удалении USB--устройств. Для этого, при подключении или удалении устройства, создается экземпляр структуры \texttt{int\_usb\_device} и помещается в список отслеживаемых устройств или удаляется из него.

В листинге~\ref{lst:int-usb-2} представлены функции для работы со списком отслеживаемых устройств.

\section{Идентификация устройства как доверенного}

Для проверки устройства необходимо проверить его идентификационные данные с данными доверенных устройств. 

В листингах~\ref{lst:usb-identify-1}---\ref{lst:usb-identify-2} представлены объявление списка доверенных устройств и функции для идентификации устройства.

\section{Обработка событий USB--устройства}
\label{sect:usb-not}

При подключении устройство добавляется в список отслеживаемых устройств. После этого происходит проверка на наличие среди отслеживаемых устройств недоверенных, и, в случае если такие были найдены, происходит отключение драйвера сети. Отключение происходит путем вызова программы \texttt{modprobe} через \texttt{usermode-helper API}.

В листинге~\ref{lst:usb-insert} представлен обработчик подключения USB--устройства.

При отключении устройство удаляется из списка отслеживаемых устройств. После этого происходит проверка на наличие среди отслеживаемых устройств недоверенных, и, в случае если такие не были найдены, происходит включение драйвера сети. Включение также происходит путем вызова программы \texttt{modprobe} через \texttt{usermode-helper API}.

В листинге~\ref{lst:usb-remove} представлен обработчик отключения USB--устройства.

\section{Уведомитель для USB--устройств}

В листинге \ref{lst:notifier-usb} представлено объявление уведомителя и его функции--обработчика.

\listingfile{netpmod.c}{notifier-usb}{C}{Уведомитель для USB--устройств}{linerange={25-30}}

В листинге~\ref{lst:notifier-usb-impl} представлено определение функции--обработчика уведомителя для USB--устройств. Функции, используемые в теле данного обработчика описаны в разделе~\ref{sect:usb-not}.

\clearpage

\listingfile{netpmod.c}{notifier-usb-impl}{C}{Уведомитель для USB--устройств. Часть 2}{linerange={239-256}}

\section{Обработка событий клавиатуры}
\label{sect:kbd-not}

Уведомители от клавиатуры поддерживают пять типов событий: KBD\_KEYCODE, KBD\_UNBOUND\_KEYCODE, KBD\_UNICODE, KBD\_KEYSYM и KBD\_POST\_KEYSYM. Каждый из обработчиков событий клавиатуры получает все пять типов событий. 

Событие KBD\_KEYSYM позволяет получить информацию о введенном сиволе из таблицы ASCII, в связи с чем будет использоваться обработчик именно этого события.

При включенном сетевом драйвере не имеет смысла обрабатывать события, поступающие от клавиатуры, в связи с чем следует проверять состояние сетевого драйвера прежде, чем выполнять последующие действия.

В листинге~\ref{lst:kbd-verify-action} представлена функция валидации событий от клавиатуры.

Еслине задан пароль для включения сетевого драйвера, следует полностью исключить возможность включения сетевого драйвера без удаления недоверенных USB--устройств. 

В листинге~\ref{lst:kbd-verify-pwd-len} представлена функция валидации пароля, указанного в параметрах загружаемого модуля.

Наконец, в листинге~\ref{lst:kbd-action} представлена функция обработки введенного символа.

\section{Уведомитель для клавиатуры}

В листинге \ref{lst:notifier-kbd} представлено объявление уведомителя и его функции--обработчика.

\listingfile{netpmod.c}{notifier-kbd}{C}{Уведомитель для клавиатуры}{linerange={32-37}}

В листинге \ref{lst:notifier-kbd-impl} представлено определение функции--обработчика уведомителя. Функции, используемые в теле данного обработчика описаны в разделе~\ref{sect:kbd-not}.

\listingfile{netpmod.c}{notifier-kbd-impl}{C}{Обработчик событий от клавиатуры}{linerange={311-326}}
     
\section{Регистрация уведомителей}

В листинге \ref{lst:mod} представлены регистрация и дерегистрация уведомителей при загрузке и удалении модуля ядра соответственно.

\listingfile{netpmod.c}{mod}{C}{Регистрация и дерегистрация уведомителей}{linerange={39-62}}