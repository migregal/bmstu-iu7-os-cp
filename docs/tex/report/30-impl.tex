\chapter{Технологический раздел}

\section{Выбор языка и среды программирования}

Разработанный модуль ядра написан на языке программирования \texttt{C}~\cite{c-language}. Выбор языка программирования \texttt{С} основан на том, что в настоящий момент большая часть исходного кода ядра Linux, его модулей и драйверов написана на данном языке~\cite{rust-in-linux}.

В качестве компилятора выбран \texttt{gcc}~\cite{gcc}. Выбор обоснован тем, что данный компилятор является предпочтительным для сборки Linux~\cite{build-linux}.

В качестве среды разработки выбрана среда \texttt{Visual Studio Code}~\cite{vscode}.

\section{Загружаемый модуль ядра}

\subsection{Уведомитель для USB--устройств}

В листинге \ref{lst:notifier-usb} представлено объявление уведомителя и его функции--обработчика.

\listingfile{netp_mod.c}{notifier-usb}{C}{Уведомитель для USB--устройств. Часть 1}{linerange={23-28}}

В листингах~\ref{lst:notifier-usb-impl}~и~\ref{lst:notifier-usb-impl-2} представлено определение функции--обработчика уведомителя для USB--устройств. Функции, используемые в теле данного обработчика описаны в разделе~\ref{sect:usb-handler-impl}.

\listingfile{netp_mod.c}{notifier-usb-impl}{C}{Уведомитель для USB--устройств. Часть 2}{linerange={202-208}}

\listingfile{netp_mod.c}{notifier-usb-impl-2}{C}{Уведомитель для USB--устройств. Часть 3}{linerange={209-219}}

\subsection{Уведомитель для клавиатуры}

В листинге \ref{lst:notifier-kbd} представлено объявление уведомителя и его функции--обработчика.

\listingfile{netp_mod.c}{notifier-kbd}{C}{Уведомитель для клавиатуры}{linerange={30-35}}

В листингах~\ref{lst:notifier-kbd-impl}~и~\ref{lst:notifier-kbd-impl-2} представлено определение функции--обработчика уведомителя. Функции, используемые в теле данного обработчика описаны в разделе~\ref{sect:kbd-handler-impl}.

\listingfile{netp_mod.c}{notifier-kbd-impl}{C}{Обработчик событий от клавиатуры}{linerange={274-281}}

\listingfile{netp_mod.c}{notifier-kbd-impl-2}{C}{Обработчик событий от клавиатуры}{linerange={282-289}}

\subsection{Регистрация уведомителей}

Реализация алгоритма из пункта~\ref{sect:lkm-design}, приведена в листинге~\ref{lst:mod}.

%В листинге \ref{lst:mod} представлены регистрация и дерегистрация уведомителей при загрузке и удалении модуля ядра соответственно.

\listingfile{netp_mod.c}{mod}{C}{Регистрация и дерегистрация уведомителей}{linerange={39-60}}

\section{Обработчик событий от USB}

\subsection{Хранение информации об отслеживаемых устройствах}

Для хранения информации об отслеживаемых устройствах объявлена структура \texttt{int\_usb\_device}, которая хранит в себе идентификационные данные устройства (\texttt{PID, VID, SERIAL}), а так же указатель на элемент списка.

Структура \texttt{int\_usb\_device}, а так же инициализация списка, в котором будут храниться данные структуры представлены в листинге~\ref{lst:int-usb}.

\listingfile{netp_mod.c}{int-usb}{C}{Структура \texttt{int\_usb\_device}}{linerange={66-81}}

Список отслеживаемых устройств должен модифицироваться при подключении и удалении USB--устройств. Для этого, при подключении или удалении устройства, создается экземпляр структуры \texttt{int\_usb\_device} и помещается в список отслеживаемых устройств или удаляется из него.

В листинге~\ref{lst:int-usb-2} представлены функции для работы со списком отслеживаемых устройств.

\clearpage

\listingfile{netp_mod.c}{int-usb-2}{C}{Функции для работы со списком устройств}{linerange={129-152}}

\subsection{Идентификация устройства как доверенного}

Для проверки устройства необходимо проверить его идентификационные данные с данными доверенных устройств. 

В листингах~\ref{lst:usb-identify-1}--\ref{lst:usb-identify-3} представлены объявление списка доверенных устройств и функции для идентификации устройства.

\listingfile{netp_mod.c}{usb-identify-1}{C}{Функции идентификации устройств. Часть 1}{linerange={83-86}}

\listingfile{netp_mod.c}{usb-identify-2}{C}{Функции идентификации устройств. Часть 2}{linerange={87-123}}

\listingfile{netp_mod.c}{usb-identify-3}{C}{Функции идентификации устройств. Часть 3}{linerange={124-127}}

\subsection{Обработка событий USB--устройства}
\label{sect:usb-handler-impl}

Реализация алгоритмов из пункта~\ref{sect:usb-handler-design}, приведена в листингах~\ref{lst:usb-insert}~и~\ref{lst:usb-remove}.

Отключение и восстановление сети происходит путем вызова программы \texttt{modprobe} через \texttt{usermode-helper API}.

\listingfile{netp_mod.c}{usb-insert}{C}{Обработчик подключения USB--устройства}{linerange={155-176}}

\clearpage

\listingfile{netp_mod.c}{usb-remove}{C}{Обработчик удаления USB--устройства}{linerange={179-199}}
    
\section{Обработчик событий от клавиатуры}

\subsection{Обработка событий клавиатуры}
\label{sect:kbd-handler-impl}

Уведомители от клавиатуры поддерживают пять типов событий: KBD\_KEYCODE, KBD\_UNBOUND\_KEYCODE, KBD\_UNICODE, KBD\_KEYSYM и KBD\_POST\_KEYSYM. Каждый из обработчиков событий клавиатуры получает все пять типов событий. 

Событие KBD\_KEYSYM позволяет получить информацию о введенном сиволе из таблицы ASCII, в связи с чем будет использоваться обработчик именно этого события.

\clearpage

\listingfile{netp_mod.c}{kbd-data}{C}{Объявление используемых значений}{linerange={225-227}}

В листинге~\ref{lst:kbd-verify-action} представлена функция валидации событий от клавиатуры.

\listingfile{netp_mod.c}{kbd-verify-action}{C}{Функция валидации события}{linerange={228-239}}

Если не задан пароль, следует полностью исключить возможность включения сети без удаления незарегистрированных USB--устройств.

В листинге~\ref{lst:kbd-verify-pwd-len} представлена функция валидации пароля, указанного в параметрах загружаемого модуля.

\listingfile{netp_mod.c}{kbd-verify-pwd-len}{C}{Функция валидации пароля}{linerange={241-251}}

Наконец, в листинге~\ref{lst:kbd-action} представлена функция обработки введенного символа.

\listingfile{netp_mod.c}{kbd-action}{C}{Функция обработки символа}{linerange={253-272}}

Реализация алгоритма из пункта~\ref{sect:kbd-handler-design}, приведена в листинге~\ref{lst:kbd-full}.

\listingfile{netp_mod.c}{kbd-full}{C}{Обработчик событий клавиатуры}{linerange={274-289}}
